
\chapter{Implementation}
\label{ch:sw}

\section{Objectives of the Implementation}
\label{ch:sw:objectives}

The implementation of the Astris voting system will be utility-first. It is intended to prove the feasibility of creating such a system and provide an insight into the practical application of the protocol. User-friendliness will not be a priority, nor efficiency, but instead correctness will be the focus. This extends to some aspects of the peer-to-peer networking such as automatic peer-discovery and black/white listing capability (although these features may be implemented in the fullness of time).

The objective of the implementation is to create a software service that can adhere to the protocol and can perform the roles of:

\begin{itemize}
    \item \textbf{Authority:} the election organiser role. Creating the initial election parameters and setup information, the pieces required to allow the trustees and registrar to perform their setup duties and to create the genesis block.
    \item \textbf{Trustee:} one of the decryption trustees. Participating in the distributed key generation algorithm in the setup phase, and performing the partial decryption in the tally phase.
    \item \textbf{Registrar:} the eligibility authority. Creating the setup parameters, keys and voter list. Running a web-server to provide a dummy authentication module to allow voters to have there voting keys signed.
    \item \textbf{Voter:} a potential voter. Creating a voting key-pair, creating the registration URL for the voter to authenticate with the Registrar and have their key signed. Adding the signed key to the chain.
    \item \textbf{Auditor:} a member of the network simply validating all blocks on the chain, and therefore all workings of the election.
\end{itemize}

The \textbf{Auditor} is actually performed by an of the other roles when they connect to the blockchain, they will all validate all the blocks.

The software should show that these functions can each be run independently, with no collusion and that the privacy of the voter is maintained.

\section{Architecture of the Software}
\label{ch:sw:architecture}

I have chosen Go (\url{https://golang.org/}) as the implementation language. This choice was for a few main reasons not least of which is that I am already familiar with the language and have implemented several small projects in it. Another good reason for the choice is that the simplicity and enforced formatting make most Go projects easily accessible to new contributors. Finally, the next biggest reason was that Go code can be compiled to \gls{wasm} easily, meaning the same cryptographic code could be used in the browser as natively.

I have chosen to use gRPC (\url{https://grpc.io}) as the inter-node communication protocol as hopefully it should make third party implementations more straightforward as they can share the protobuf definitions.

The core blockchain will be backed by a SQLite3 database for simplicity and ease of use, although the interface will be abstract to allow other storage backends should we want to. SQLite3 remains an incredibly robust, dependable and versatile format.

The payload for the blocks will be represented as arbitrary byte arrays at the blockchain level, and the extra validation of the Astris protocol will be performed at a higher level. This will keep the block exchange and networking code isolated from the Astris protocol logic.

The application will be constructed as a library, and the various CLI invocations for the different roles will simply call different parts of the inner library to perform their tasks.

The code for this project is versioned with the Git version control software and a copy of the repository is located on GitHub. The software may well see further development after this report is completed so for reference, the version at the time of publishing is available at \url{\astrisrepo /tree/ \astrishash}.

\section{Notes on Implementation}
\label{ch:sw:notes}

\todo{
    What I learned during the implementation phase. What was challenging. \\
    What was easy. How well it went. etc ... \\

    Mainly that there is an enormous amount of work involved, the blockchain and crypto are a small part of the overall project, the finer grained detail of the peer-to-peer gossiping and managing the chain where much more difficult than anticipated.

}
