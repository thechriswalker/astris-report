
\chapter{E-Voting}
\label{ch:ev}

Electronic Voting by definition covers any electronically assisted form of voting. This could be electronically counted paper votes, or electronic voting machines at a voting booth. However, we will be considering another option, of fully electronic voting where the systems are accessible over the Internet --- similar to the general usage of the term \emph{email}.

\section{Specific Requirements for E-Voting}
\label{ch:ev:specific}

The main difference between in-person voting and remote voting from the point of view of security is the lack of a secure voting booth. Such a booth is often listed as a requirement of in-person voting systems, as an assumption to allow enforcement of the eligibility and secrecy properties. That is a secure voting booth is one where the voter is authenticated on the way in and then their privacy whilst inside casting their vote is guaranteed. While a remote voting system does not preclude the use of such a booth, such a system loses a lot of it's perceived benefits by mandating the use of one. Indeed, it may not be immediately obvious why such a system would have any benefits over a traditional in-person election.

It could be argued that a seemingly obvious benefit ---that people can vote without having to travel to a secure booth--- is not really a benefit at all as we lose the security properties guaranteed by such a booth and the trade-off of convenience over security is not worth it. On the other hand remote voting does indeed have some beneficial implications. It would lower the cost of an election as there would be no need to hire secure locations or the people to run them. There would be no need for transportation of the ballots for tallying and no manual tallying process --- both lowering costs and reducing the likelihood of human errors in the tallying. Furthermore, the convenience of not having to travel to a booth is also an enabler for those who would otherwise be unable to make it to a booth, potentially increasing the election turnout.

These benefits need to be weighed against the potential security downsides. It is worth noting that there is one insurmountable problem with remote voting which is that there is no possibility to satisfy the coercion-resistance property. This has been addressed in \cite{juelsCoercionResistantElectronicElections2002} and under their definitions the property we are discussing here is \emph{simulation attack} in which the adversary acquires the means to vote on behalf of the user. No matter how secure the voting scheme is from a technical point of view, allowing voting from an arbitrary location via the Internet means that a coercer could stand over the voter during the casting process, a situation which technically indistinguishable from the voter voting alone. In \cite{juelsCoercionResistantElectronicElections2002} the solution is based on the assumption that at some point the voter will be able to vote unhindered and that that vote will be the one counted, any coerced votes would not count and additionally that the unhindered vote could not be linked by the adversary to the voter. This would allow the voter to \emph{pretend} to cooperate with the coercer and separately vote legitimately without the coercer's knowledge (and hence no fear of retribution). The scheme proposed imposes an overhead which is quadratic with respect to the number of voters, which is impractical for large scale elections. Further work has been done to attempt to reduce this overhead to be linear \cite{weberCoercionResistantElectronicElections2007a}, but even so, the overhard is burdensome for large scale national elections and the coercion resistance is still predicated on the fact that the voter will have an opportunity to act unhindered.

Due to these trade-offs we consider how some security properties mentioned in \autoref{ch:req:sec} that take on more or less importance in a remote system and some new security properties that can enhance a remote voting system.

The \textbf{Secrecy} property in a remote voting system becomes more important as now the cast ballot must be sent over a public channel from the voter to the trustees counting the votes. In some respects this is the similar to a paper ballot where, despite the ballot being placed in the ballot box by the voter, the ballot itself must not reveal the voter's identity or the trustees could match votes to voters. However, given that the vote is created by the voter remotely, there is also the transmission of the vote from the voter to the trustees ---analogous to putting a paper vote into a ballot box--- which is now performed over the public Internet, rather than within a secure voting booth.

The \textbf{Robustness} property in a remote voting system becomes more important as by definition the system will have computer systems at various boundaries which by necessity will be accessible over the public Internet. \Gls{dos} attacks against parts of the system will become a valid threat against the system continuing to function correctly. The more entities are in play in the system ---e.g. the Authority, Registrar, Trustees, Voter, Bulletin Board--- the more opportunity for an adversary to disrupt the communication channels between them. This works both ways as with fewer single entities responsible for each function in the system, the more work an adversary will have to disrupt \emph{all} of them. Therefore, remote voting systems must consider single points of failure and the public accessible nature of the system very carefully.

The \textbf{Eligibility} property is vital in all voting systems. Great care must be taken in remote voting systems to ensure eligibility, as we must authenticate the voters remotely and the protocol used must be secure. A failure in the authentication protocol would undermine the integrity of the election. This area is a major cause for concern as eligibility and the one-voter-one-vote semantics that make the election function are at risk. The method of authentication in specifications of votings system is mostly unspecified but defined to be secure as an assumption. I believe this to be a reasonable assumption as the exact technology used to provide the authentication may change without affecting the core properties of the voting system. However, all the systems looked at in this report fall into two categories: those that authenticate early with a ``registration'' phase and those that authenticate \emph{just-in-time} --- at the point of casting a vote.

Those that authenticate early predominantly use the registration phase to use the authentication process to create election specific credentials. This has the benefit that the credentials are shorter-lived, as they only apply to a single election, and that the Registrar has no further involvement during the rest of the election process. By no longer participating in the election after the initial phase, an adversarial Registrar could not create fictional voters to ballot stuff, nor register on behalf of users who did not register themselves to \emph{steal} their vote. This process provides eligibility verifiability, where we (the public) can tell that a voter, and the vote, is eligible.


The \textbf{Receipt-Freeness} property is primarily a factor adding to \emph{coercion-resistance} and a protection against vote-buying. With the fact that coercion-resistance is practically impossible for remote voting systems, it might be tempting to write-off receipt-freeness as unimportant. However, we can still counter all forms of vote-buying and coercion that do not involve physical presence ---which is expensive and logistically difficult on a large scale--- by ensuring receipt-freeness. If there was no receipt-freeness then a voter could use that receipt to prove to a coercer or vote-buyer the way they voted, or make such a proof public. With receipt-freeness this is not possible and without proof, it is unlikely that a coercer or vote-buyer would be willing to believe the voter. Due to this I believe that receipt-freeness is just as important in a remote system as in an in-person system.

\textbf{Real-Time Universal Verifiability} is the property of a system that allows the property of universal verifiability to be conducted in real-time during the election. That is the bulletin board can be monitored in real-time and modifications observed. This would allow an external Auditor to be sure that the data on the bulletin board are not changed in a way that the voting system disallows, and such illegal modifications could be proven to have been made. This property enhances the universal verifiability property with time constraints. No party would be able to, with sufficient ability to break any cryptographic protocols, rewrite the bulletin board data to meet their goals while still validating under the rules of the voting system. They could only attack the system during the election itself, with all data making it onto the bulletin board.

\textbf{Eternal Secrecy} is a property that states that the connection between the voters identity and the vote they cast should never be able to be revealed. This means that any data released on the bulletin board, or during the voting process, should never reveal the link between the voter and their vote even if all the cryptographic protocols used are broken. This means that, for example, a voting system relying on a cryptosystem that can be broken with a quantum computing algorithm ---while secure today--- would not be secure in a future where such quantum computers exist. In that future, all the protections from the cryptosystem would be void. We need to ensure that the link between voter and vote is not compromised even then. Note that such a break before or during an election would break the \emph{fairness} property, as the broken cryptosystem would reveal the state of the election before all votes were cast. So we will assume that this property is concerned with the future, after the end of the election.


\textbf{Software Independence} is a property that enforces that any undetected changes or errors in the software cannot lead to undetected changes or errors in the result. This property is discussed extensively in \cite{rivestNotionSoftwareIndependence2008} and is extremely important from a trust point of view. The concept is that a bug ---or indeed an actively created malicious piece of code--- in the software for any stage in the election should not be able to produce data that any other correct software will accept as valid. This is important for trust because it means that even the software running the election cannot violate the principles of the protocol knowingly or unknowingly without the error be noticed.

\section{Existing Systems}
\label{ch:ev:existing}

There are many practical and theoretical remote voting systems --- more than could be covered here. Therefore, we look at a selection that use a variety of technologies and techniques to achieve their goals, investigate the security properties they have and discuss the trust implication of their structure. See \autoref{table:voting-system-props} for a table of comparison of the various systems and the properties they achieve. This is not as binary as it may appear, as many of the properties are satisfied with a given assumption of trust. Therefore, the table shows an indication of the level of trust required to achieve the property, if it can be achieved at all.

\subsection{Helios}

Helios Voting is an open-source software implementation of a voting system designed by Ben Adida \cite{adidaHeliosWebBasedOpenAudit2008}. It has progressed significantly since the conference in 2008, dropping the initial mix-net approach in favour of the use of homomorphic encryption. The project has always focussed on correctness and auditability and openly claims that is only suitable for \emph{low-coercion} elections \cite{HeliosVotingFAQ}. That is, Helios makes no attempt to provide coercion-resistance.

The current version of the Helios protocol is V3 and this is the version we shall consider. The system is created as a Python program accessible via a web interface using the PostgreSQL database for data storage. The system is shared tenancy, so all users' data are stored in a single, shared database; except of course that being open-source, anyone is free to run their own private server.

The basic flow of an election is that the creator lists candidates and defines the number of candidates that may be voted for. Additionally, a list of eligible voters may be created or the system may allow anyone to vote. A public/secret keypair is generated for the election and the election can have a number of trustees who each receive a portion of a private key enabling the decryption of votes to be a group operation. Voters cast a vote for each candidate where each is either a 0 or 1 and there are only a predetermined number of 1's allowed. The vote is encrypted with the public key, and the cryptosystem used (Exponential ElGamal) is additively homomorphic. In this way votes may be tallied while still encrypted and the final tally decrypted separately by the group of trustees. This is a simplification for brevity, the full protocol spec is available \cite{HeliosHeliosV3}. Notably this misses out the digital signatures and \glsplural{zkp} used in the protocol.

Helios provides individual and universal verifiability through the publishing of the encrypted votes, along with enough proof to show that voters have voted within the bounds set in the election. The use of \glsplural{zkp} ensures that the votes conform to the correct structure and the homomorphic nature allows them to be tallied without decryption. This allows the encrypted vote data to be published in such a way that anyone can verify a specific vote is present and that the tally is correct. The vote data is available throughout the election process, so real-time universal verifiability is possible. The decryption of the tally also uses proofs based on verifiable decryption, so without the key we can still be sure the tally was decrypted correctly providing the full universal verifiability. Note that this does not satisfy our definition of auditability which enforces that we must be able to discern the correct behaviour during and after the election and detect tampering. During the election the cast votes may be listed, but the onus is on the observer to prove that the list has been tampered with, Helios does not provide such. Once the result is published, the individual and universal verifiability properties combined allow any voter to check that their vote was included in that tally. Therefore, it stands to reason that an adversary, even with full access to the server running the election, could not remove votes without chance of detection. However, the Helios protocol does not forbid such an adversary the ability to create simulated votes for the absentee voters before calculating the totals. This implies that a high level of trust is required in the authority running the Helios server.

By default, the Helios elections identify voters with \gls{pii} --- in this case their email address. While this does not break the secrecy property directly, eternal secrecy is broken. Helios does have the ability to only identity voters by an alias which is a unique identifier not related to the individual and unique in each election. Provided the link between the voter and the alias is removed after the tallying, the eventual breaking of the cryptography would still not reveal the identity of the voter. Again, a dishonest server administrator could obtain that information from the server before it was removed, which implies a level of trust in the authority running the server, but not more so than already existed.

The voters are authenticated either by a password which the Helios system knows, or via an external authentication solution such as OpenID Connect. The exact method of authentication is configurable per election. This enforces the eligibility property with the exceptions of dishonest authentication providers ---registrars--- or dishonest server administrators. The result of the authentication of the user is not propagated into the vote, so with direct access to the database spurious votes could be cast.

The votes in Helios are encrypted by the client, before posting to the server. This means we do not need to trust the server with unencrypted votes. The encryption process itself however requires the client to generate some randomness to use during encryption and subsequently destroy it. This randomness could be used to prove which candidate the vote was for and so breaks the receipt-freeness property.

In summary, Helios' main weakness from a trust perspective is that much of its security is based on the security of the server running the software. This is often an acceptable assumption, given the specific threat model for the election being held.

\subsection{Belenios}

Belenios was heavy influenced by Helios, and the original implementation was indeed a fork of the Helios source code adding a distributed key setup for the trustees \cite{cortierBeleniosSimplePrivate2019}. It was subsequently re-implemented from scratch, but the protocol shows many similarities.

A variant, BeleniosRF, enhances the protocol to provide receipt-freeness by use of a re-randomization step. The votes are encrypted and signed using the ElGamal cryptosystem, and a random number is required for each encryption. The knowledge of that number can prove your vote and so provide a receipt. The randomness is generated by the client, and so could be kept to provide that receipt to the voter. In BeleniosRF the encrypted vote and signature can be re-randomized at the voting server and a new valid signature created without the signing key. The voting server does not know the original vote as it is encrypted and the final cast vote is encrypted with randomness that the voter does not know, hence the protocol is receipt-free.

A second variant, BeleniosVS goes further and provides a mechanism by which the vote does not even need to trust their own client. This is achieved by the vote being provided with a voting sheet ahead of time with the encrypted and signed votes already prepared. This sheet would also contain the randomness used for each encryption, so the voter can verify the encrypted values are correct. Combined with the BeleniosRF system for re-randomization, the vote is receipt-free and the user does not have to trust their own device. Given a compromised device used for voting, the device could prevent submission of the vote but the voter would be able to confirm that. However, the device would not be able to create any vote on behalf of the user as the user never provides any secrets to it. This system seems like it would be an excellent fit for an election with a secure voting booth. The booth provides the guaranteed isolation and the use of pre-prepared encryptions of votes means the voter need not trust the devices used for the voting.

We will consider this second variant, BeleniosVS, as it appears to be an evolution of the original. Being based on Helios, we will focus on the differences. BeleniosVS uses distributed key setup for the trustee keys that ensure that no entity ever sees the full decryption key at any point during the process. The key setup is also a threshold based system, meaning that only a subset of $M$ out of the $N$ trustees are required to perform the decryption. This has two effects. Firstly, given that $x < M$ trustees are dishonest, the cast votes will remain secret, irrespective of the security of the server running the software or its database. Secondly, it means that $N-M$ trustees can decide not to participate in the tallying and the result can still be calculated. There is a natural trade-off here between the additional robustness of having $N-M$ large, the additional confidence of secrecy by having $M$ large and the practical aspects of having $N$ large.

\subsection{Estonian i-Voting: IVXV}
\label{ch:ev:existing:ivoting}

Estonia has embraced remote voting and the i-Voting system known as IVXV has been used to run state level parliamentary elections since 2005. The description of the system and a limited amount of the source code used to run it is publicly available online\footnote{Source code at \surl{https://www.valimised.ee/en/internet-voting/documents-about-internet-voting}}. The system has had much attention since its inception as the primary example of binding governmental elections held remotely, such as \cite{OSCEODIHRElection2019,SecurityAnalysisEstonian}. Four reports from the Office for Democratic Institutions and Human Rights (OSCE/ODIHR), most recently in 2019 have observed the election processes and commented on potential security issues. These reports are designed to continually assess the current state of the system and its reaction to previous recommendations. Judging from the statistics regarding the proportion of votes that come from remote voters, the system is widely popular amongst the electorate: in the 2019 election 43.8\% of votes were cast remotely according to \cite{OSCEODIHRElection2019}.

The system allows remote votes to be cast for a time period before the physical casts. Voters may cast as many electronic votes as they wish, with a ``last-vote-counts'' mechanism for differentiating the valid vote. Finally, voters may also vote via the in-person methods with this vote overriding any previous electronic votes.

The system is based on nationally run \gls{pki} for authentication of voters --- a system which is widely used in Estonia to authenticate citizens and create legally binding signatures. The identity system provides the voter with a private key that can be used to create such signatures. There is also a key generated specifically for the election. There are three distinct entities involved in the process, the Collector, the Processor and the Tallier.

Votes are encrypted with the election-specific key, then the encrypted vote is signed with the voter's personal key and passed to the Collector, who does not have access to the private key for decrypting the votes. Once all votes have been cast, the Collector passes all votes to the Processor who verifies all the signatures and selects the correct single vote from any voter that has cast multiple, ensuring the ``last-vote-counts'' semantics. The Processor cannot decrypt the votes, so although it can match encrypted votes to voters, it cannot infer the voters' choices. The Processor then strips the signatures from the valid votes and passes them in bulk to the Tallier. As such, the Tallier does not have the ability to match encrypted votes to the voter that cast them, but the Tallier can decrypt the votes and so produce a final tally.

The observations of the ODIHR in \cite{OSCEODIHRElection2019} stated that the system was improved compared to the time of the previous report, however certain internal data ---considered \emph{private} but available to those with privileged access to the internal systems--- contained enough information to break secrecy, when a voter had made their \emph{receipt} public. This receipt is intended to be solely prove that a vote cast was counted, satisfying \emph{individual verifiability}, but in this case would allow the entity with internal access the ability to reveal the voter's vote. This class of threat, that of internal users with elevated privileges accessing raw data presumed secret, exists in other remote voting systems and its severity may only be ascertained by the threat model and risk assessment of the specific election. In the case of the IVXV, these are binding national governmental elections and so the stakes are high.

On the other hand, the role of Collector, Processor and Tallier are provided by independent entities which reduces the chances of collaboration which would also defeat the secrecy of the election. This is only a reduction of chances, rather than a full mitigation as an adversary would simply need to subvert all three entities, rather than just one to be in a position to attack the election. This would objectively be more difficult, but not impossible.

Another area in which the report was recommended change was that the system was not \emph{software independent} stating: ``meaning that errors in its components may cause undetected errors in the election results, and it is potentially vulnerable to internal attacks and to allegations of cyber attacks''. This last point is an interesting one, and corroborates a claim that the author would make: that the potential for vulnerability, whether exploited or not, leaves room for plausible allegations which can undermine trust.

Despite the criticism and the potential security flaws, the published statistics from the Estonian elections show that many voters can and do overcome the trust requirements for submitting their votes remotely and that the number of voters willing to do so is increasing over time. It is the author's belief that the widespread use of digital identity in Estonian governmental processes which has been happening for many years--- has given the citizens enough confidence in the use of this identity for other legally binding commitments that they are willing to extend the same level of trust to the national voting system. As long as the system continues to work effective without major cyberattacks or public failings, the author expect that confidence to rise. However, a major public failing of the system would likely have severe consequences on public trust of the digital identities. This make the continued success of the system balance on a knife edge, with public support predicated on the lack of incident, but incident always possible.


\subsection{Bitcoin Voting Proposal -- Zhao and Chan}

This section looks at the system proposed in \cite{zhaoHowVotePrivately2016} which creates a voting system on top of the Bitcoin cryptocurrency. The system is limited to a two candidate election, with each voter able to vote for one candidate or the other, without the possibility of abstention. There are further practically issues in this system that make it unsuitable for real-world usage, such as the fact that it requires real bitcoin transactions that are not free. Whilst the proposal itself used 1 Bitcoin as the base pricing unit ---which would be an inconceivable amount with the current price of Bitcoin--- but even using a smaller base, the real-world costs would be problematic. To core of the proposal is that the voters transfer bitcoin to the winner.

The system describes two possible voting methods. The first behaves in such a way that should the protocol fail and the election not be satisfactorily concluded by a certain time, then the voter can receive the deposited amount back by creating a refund transaction. This first method however requires that each subsequent voter commits a larger amount of bitcoin in their deposit transaction. The second allows the voter to get back a deposit, but not the total cost of voting. Indeed, in either situation the cost of transactions themselves is not taken into account and every transaction has a non-zero cost in and of itself that goes to the miner of the block. While both these methods incentivize good behaviour of voters by imposing real-world costs for deviating from the protocol, the costs themselves impose a barrier to entry.

Another practicality issue is that the all the voters must contribute to the protocol in the correct stages and in the correct order, or the protocol fails and the refunds can be claimed but no winner decided. This make the process fragile, breaking the \emph{robustness} property.

Therefore, while much of the algorithmic processes within the scheme are sound, there is an extremely high opportunity for any single voter to completely void the entire election by withholding a vote. This places an extremely high level trust required for each participant to give to each other participant.

\subsection{ZCash Voting Proposal - Tarasov and Tewari}

This section looks at the system proposed in \cite{tarasovInternetVotingUsing2017} which uses the ZCash cryptocurrency as a base for a voting system. ZCash is a cryptocurrency system which supports two address types. The first: a \emph{z-address} is which is anonymous and transactions to or from them cannot be linked back to any other transaction. The second is a \emph{t-address} which is functionally similar to a bitcoin wallet address, and as such the movement of coins to or from these addresses can be publicly discerned.

The proposal uses a registration phase and the Registrar entity also plays the roles of the Election Authority. Once a voter has registered, their data is held in an unspecified datastore by the Registrar/Authority entity. On the \emph{opening} of the vote, the Registrar/Authority sends invitation links by email to the voters. The voter's then provide a \emph{t-address} to which a ``vote token'' is provided. This must come from some pre-existing source of coins and the state of which voters have been issued tokens must be tracked by the Registrar/Authority entity. This source is called the \emph{ZEC pool}.

Once the voter has their ``vote-token'' there are two variants described for the voting process.  The first variant sees the candidate using a \emph{z-address}. This means that the transaction is anonymous in that no one can introspect the details of the vote. However, it also ---by design--- means that the vote is now unable to verify that their token is part of the system; they cannot trace their own vote has been counted.

The second variant has the candidate using a \emph{t-address}. In this variant the token balance is discernable to all parties at all times, and the link between token and vote is preserved. While this latter property would allow the voter to verify their vote, it would also allow the authority to match the vote to the voter.

The final tally and audit stage of the process sees the candidates transfer the balance of the tokens they have back to the \emph{ZEC pool}. The number of tokens that came from each candidate are counted and made public. The Registrar/Authority is tasked ---and \emph{trusted} with performing the task of verification that the count of votes balances with all previous data.

The proposal itself contains a section on the security considerations, which reveals some areas of trust that exist within the protocol. The first voting variant, using candidate \emph{z-addresses} have no way of tracking the balance of votes that a candidate has received. This enables the \emph{fairness} property as no indication of the ongoing status of the election is revealed until the end. However, it does mean that we rely on the candidate to transfer the full contents of the address back to the pool for audit. An untrustworthy candidate may decide to wait until the other candidates have transferred their tokens and if they have not won the election ---a fact that they can verify--- they could underreport their votes with two consequences. Firstly, they could keep the other tokens, which have a non-negligible real-world monetary value and secondly, the election token balances would not add up and the election would be void. This latter situation could mean a re-vote which may be in the untrustworthy candidate's favour, or simply a denial of the result for the otherwise victorious candidate.

In the second voting variant, the underreporting of vote tokens would be noticeable as the candidate wallet is a \emph{t-address} and could be directly linked to the guilty candidate. Whether this would void the election or simply disqualify the candidate is not specified. However, in the second system the \emph{fairness} property has been broken, which in the author's opinion is enough to disqualify this proposal as a valid voting system.

A great deal of the state of the system is kept private by the Registrar/Authority yet required to be used in later stages of the election process. The integrity and authenticity of this state is not considered and there are no details to how it may be stored and validated. This would break the property of software independence, as the vote tokens and transactions may be on the blockchain the auxiliary state is not.

In summary, this proposal despite running on a blockchain requires an extremely high level of trust from its participants. The number of ways an adversarial entity ---whether voter, registrar, authority, or candidate--- could disrupt the system means that it is not \emph{robust} unless all parties are honest. While that does not prevent the system from functioning, it does make it impractical and imposes high trust requirements from voters who know that they must trust many entities to behave correctly.

\subsection{Scalable Open Vote Network}

This section looks at the system proposed in \cite{seifelnasrScalableOpenVoteNetwork2020} which itself is an evolution of \cite{mccorrySmartContractBoardroom2017} and uses smart contracts on the Ethereum blockchain as a basis for a voting. The system is limited to a binary vote i.e. the decision between one of two options. The smart contracts used ensure eligibility, the timing of the stages of the election and verify the \gls{zkp} that the vote is indeed a 0 or 1. The system addresses the issue of cost in the complexity of the smart contracts by making heavy use of Merkle trees over much of the computation, which is actually performed off chain. The computation must be \emph{disputed} if deemed invalid and the dispute process involves trigger a smart contract which performs the disputed step and will detect the dishonest behaviour or reject the transaction if the computation was actually valid. Note that this step will cost the disputer, as they must pay for the gas to perform the transaction.

To start an election the organizer creates a Merkle tree of the eligible voters which is used later to assess eligibility. The root of the tree will be part of the smart contract. Along with the timing data for the election and a deposit amount, the smart contract is deployed.

Next all voters register a voting keypair and \gls{zkp} of knowledge of the secret key. A Merkle proof of membership is constructed (which is not a fixed size, and grows with logarithmically with the number of leaves). These data along with the deposit amount are added to the chain via the smart contract. The contract will accept the voter if the timing is correct, and that the \gls{zkp} of knowledge and Merkle proof of membership are both valid.

The encryption technique used for concealing the votes is interesting. It requires using all the public keys of all registered voters, so that when \emph{all} the results are multiplied, the final result $p$ reveals the sum of the votes (it is actually the exponential, but we can brute force $log_g(p)$). At each stage of the multiplication, a leaf on a Merkle tree is created, at the end the final result and the Merkle root and stored in the contract by the organizer. The full content of the Merkle tree must be published somewhere accessible to the potential disputers.

The disputers can verify the merkle tree, and on finding an error can trigger a contract that will check the calculation and judge the dispute. If in favour of the disputer, the disputer is rewarded with the value of the deposit. After the dispute, irrespective of the outcome, all honest participants may reclaim their deposit (note a successful disputer ends up ahead one deposit, less the cost of the dispute). The organizer is only able to retrieve their deposit if no successful dispute was made.

We can inspect source code of the smart contract to ensure it will behave legitimately, so all aspects relating to the execution of smart contracts can be considered trustworthy.

The organizer in this protocol has no clear benefit to acting dishonestly, in all likelihood it will cost them the deposit amount. Even if no-one disputes in the allotted timeframe, the tally can be recomputed at any stage and the calculation show incorrect, likely nullifying any outcome from the election. However, a voter may throw election very easily by simply not casting a vote. The encryption and tallying only work if all registered voters actually cast a vote. A single abstention would render the election void.




\subsection{LWE Based proposal - Chillotti}

This section looks at the system proposed in \cite{chillottiHomomorphicLWEBased} which uses a \emph{post-quantum} homomorphic encryption system as the base for a voting protocol.

\todo{
    Post-Quantum, but otherwise, the voting is basically the same model as any homomorphic encryption based system.
}

\subsection{Vote-SAVER - Lee}

This section looks at the system proposed in \cite{leeSAVERSNARKfriendlyAdditivelyhomomorphic2019} which uses the Ethereum blockchain with a novel zero-knowledge proof system as the basis for a voting protocol.

\todo{
    This is interesting. based on a zkSNARK and smart contracts \\
    another Ethereum based system.
}

\subsection{Generic Cryptocurrency Based Voting - Yu}

This section looks at the system proposed in \cite{yuPlatformindependentSecureBlockchainBased2018} which describes a voting system that could run on any cryptocurrency.

\todo{
    Use of Linkable Ring Signatures for scalability (keeping signatures short)
}

\subsection{Trustless - Gajek}

This section looks at the system proposed in \cite{gajekTrustlessCensorshipResilientScalable2019} which focuses on producing a zero-trust protocol.

\todo{
    High claims of zero-trust \\
    Solves the privacy/eligibility in zero trust with a distributed ``off-chain oracle'' system. \\
    proptyped on HyperLedger \\
    \textbf{I have to justify why I haven't based everything on this.}
}

% for the trust on the table.
\newcommand*\YES{}
\newcommand*\NO{}

\newcommand*\NP{$n$}
\newcommand*\ALL{$Y$}
\newcommand*\SERVER{$Y_s$}
\newcommand*\REG{$Y_r$}
\newcommand*\SERVERREG{$Y_{rs}$}
\newcommand*\AUTH{$Y_a$}


\begin{table}[h]
    \centering
    \begin{tabular}{|Sr c c c c c c c c c c c c|}
        \hline
        Voting System                 & \rot{Correctness} & \rot{Secrecy} & \rot{Eligibility} & \rot{Fairness} & \rot{Robustness} & \rot{Receipt Freeness} & \rot{Coercion Resistance} & \rot{Individual Verifiability} & \rot{Universal Verifiability} & \rot{Auditability} & \rot{Real-time Universal Verifiability } & \rot{Eternal Secrecy} \\

        \hline\hline

        \textbf{Helios}               & \ALL              & \SERVER       & \SERVERREG        & \SERVER        & \NP              & \NP                    & \NP                       & \ALL                           & \ALL                          & \NP                & \ALL                                     & \SERVER               \\
        \hline
        \textbf{BeleniosVS}           & \YES              & \YES          & \YES              & \NO            & \NO              & \YES                   & \YES                      & \YES                           & \NO                           & \YES               & \NO                                      & \YES                  \\
        \hline
        \textbf{i-Voting}             & \YES              & \YES          & \YES              & \NO            & \NO              & \YES                   & \YES                      & \YES                           & \NO                           & \YES               & \NO                                      & \YES                  \\
        \hline
        \textbf{Zhao-Bitcoin}         & \YES              & \YES          & \YES              & \NO            & \NO              & \YES                   & \YES                      & \YES                           & \NO                           & \YES               & \NO                                      & \YES                  \\
        \hline
        \textbf{Tarasov-ZCash}        & \YES              & \YES          & \YES              & \NO            & \NO              & \YES                   & \YES                      & \YES                           & \NO                           & \YES               & \NO                                      & \YES                  \\
        \hline
        \textbf{Seiflelnasr-Ethereum} & \YES              & \YES          & \YES              & \NO            & \NO              & \YES                   & \YES                      & \YES                           & \NO                           & \YES               & \NO                                      & \YES                  \\
        \hline
        \textbf{Chilloti-LWE}         & \YES              & \YES          & \YES              & \NO            & \NO              & \YES                   & \YES                      & \YES                           & \NO                           & \YES               & \NO                                      & \YES                  \\
        \hline
        \textbf{Yu-Cryptocurrency}    & \YES              & \YES          & \YES              & \NO            & \NO              & \YES                   & \YES                      & \YES                           & \NO                           & \YES               & \NO                                      & \YES                  \\
        \hline
        \textbf{Trustless-Gajek}      & \YES              & \YES          & \YES              & \NO            & \NO              & \YES                   & \YES                      & \YES                           & \NO                           & \YES               & \NO                                      & \YES                  \\
        \hline
    \end{tabular}



    {\raggedright
    \vspace{0.25em}
    \footnotesize{
        \begin{tabular}{r l}
            \textbf{\NP}     & Not provided.                                            \\
            \textbf{\ALL}    & Provided under all trust assumptions.                    \\
            \textbf{\SERVER} & Provided assuming the server operator is trustworthy.    \\
            \textbf{\AUTH}   & Provided assuming the election Authority is trustworthy. \\
            \textbf{\REG}    & Provided assuming the Registrar is trustworthy.          \\
        \end{tabular}
    }
    }

    \caption{Table of Voting Systems, Security Properties and Trust Requirements}
    \label{table:voting-system-props}
\end{table}

