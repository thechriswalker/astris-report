
\chapter{Proposal for an Improved Protocol}
\label{ch:astris}

\section{Objectives and Non-Objectives of the Protocol}
\label{ch:astris:aims}

\todo{
  Which aspects I intend to improve, and what assumptions/axioms I will make.

  - removal of single points of failure \\
  - tradeoff between requiring all parties to collude vs resilience
}

\todo{explain the security requirements we aim to achieve}

\section{Protocol Overview}
\label{ch:astris:overview}

\todo{Brief outline of the basics of the protocol, with the phases mentioned.

  The protocol bears a distinct resemblance to the Helios v4 protocol. This is because Helios already contained a good amount of the privacy goals and the encryption protocols are far simpler than alternatives, whilst still being extensible into multi-party setups for shared-nothing initialization.

  We begin with the trustees mutually defining the parameters for the election, candidates, time periods, etc. The trustees also each generate their part of the multi-party key setup based on \cite{petersen_distributed_1992}.

  This means no entity has the private key and dependent on the setup $K$ of $N$ (where $K <= N$) trustees are required for decryption. $K = N$ gives at least one honest trustee keeps the whole thing honest, but one lost key ruins the entire election. The appropriate trade-off between security and robustness must be made on a case by case basis.

  One of the parameters chosen will be the eligibility authority (registrar) who will oversee the eligibility. This is problematic single point of trust. In a governmental
}



\section{Protocol Detail}
\label{ch:astris:detail}

\todo{How the phases fit to together and exactly what data is gathered at each step}

\subsection{Setup Phase}
\label{ch:astris:detail:setup}

\todo{What data is recorded onto the chain upfront?}

\subsection{Voting Phase}
\label{ch:astris:detail:vote}

\todo{
  How are voters authenticated, how are their votes secured, how they are verified. \\
  Are multiple votes allowed? last vote wins?
}

\subsection{Tallying Phase}
\label{ch:astris:detail:tally}

\todo{What is needed to be done to tally the votes, how does this not break the confidentiality of the voter?}

\subsection{Verification Phase}
\label{ch:astris:detail:verify}

\todo{How can an observer who has been part of the chain verify the result? Can an outsider that only see the final chain be convinced it is valid?}
