
\chapter{Helios in Detail}
\label{ch:helios}

\section{Overview and Aims of the Helios Project}
\label{ch:helios:aims}

Helios Voting is an open-source software project. The project's stated goals are to provide a private, verifiable and proven election system over the internet. The maintainers openly admit that coercion resistance cannot be achieved with remote voting and as such the system should not be used for large scale elections where the stakes are high and the risk of coercion is a valid concern \cite{HeliosVotingFAQ}. Version 3 of the Helios system was adopted by the \gls{iacr} \cite{HeliosCryptographers2010} in 2010 to facilitate their internal voting needs.

Version 3 of the protocol is the current in-use version, while version 4 is the next iteration and not yet released. The protocols are very similar and so we will focus on version 3. The major difference between the previous versions (1 and 2) of the Helios protocol is the move to a homomorphic, verifiable encryption to provide secrecy and verifiability where previously a mix-net construction was used.

Helios has been chosen for analysis due to its open-source nature, allowing full analysis of not only the published protocol but also the source code controlling the system


\todo{What the maintainers of Helios actually want. Anti-goals, etc...}


\section{Helios Protocol (v3)}
\label{ch:helios:v3}



\todo{The latest proposal spec info}

\section{Trust Analysis of the Helios v4 Protocol}
\label{ch:helios:trust}

\todo{Where the trust remains in the helios protocol}
