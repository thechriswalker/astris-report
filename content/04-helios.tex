
\chapter{Helios in Detail}
\label{ch:helios}

\section{Overview and Aims of the Helios Project}
\label{ch:helios:aims}

Helios Voting is an open-source software project. The project's stated goals are to provide a private, verifiable and proven election system over the internet. The maintainers openly admit that coercion resistance cannot be achieved with remote voting and as such the system should not be used for large scale elections where the stakes are high and the risk of coercion is a valid concern \cite{HeliosVotingFAQ}. Version 3 of the Helios system was adopted by the \gls{iacr} \cite{HeliosCryptographers2010} in 2010 to facilitate their internal voting needs.

Version 3 of the protocol is the current in-use version, while version 4 is the next iteration and not yet released. The protocols are very similar, and so we will focus on version 3. The major difference between the previous versions (1 and 2) of the Helios protocol is the move to a homomorphic, verifiable encryption to provide secrecy and verifiability where previously a mix-net construction was used.

Helios has been chosen for analysis due to its open-source nature, allowing full analysis of not only the published protocol but also the source code controlling the system.


\section{Helios v3 Protocol}
\label{ch:helios:v3}



\todo{Helios website provide the verification protocol and the source code contains all the missing pieces to describe and explain the why and how of this.}

\section{Trust Analysis of the Helios v3 Protocol}
\label{ch:helios:trust}

\todo{Where the trust remains in the helios protocol

    They freely admit to no coercion-resistance. but how does the system manage trust.

    There are trustees for each election that hold a shared private key - double check how the key is generated - from memory it was on the helios server, so even if it discards it, it knew it.

    The other aspect to explore is the authentication mechanism, which in helios is pluggable, but the provided interfaces? email and openid. They provide a reasonable avenue for attack - at least in theory.
}
