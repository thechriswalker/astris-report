\chapter{Introduction}
\label{ch:intro}

\section{Motivation}
\label{ch:intro:motiv}

The author is a software engineer and very interested in the role of trust in systems. This report is the culmination of my exploration of that space combined with the unique and interesting properties required from e-voting systems.

\todo{Is this what I should have in the motivation section? It seems a bit flimsy}

\section{Objectives}
\label{ch:intro:object}

This report aims to:

\begin{enumerate}
    \item Describe the security properties desirable in e-voting schemes.
    \item Analyse existing e-voting schemes, their security properties and identify areas that require significant trust from the user.
    \item Introduce the technologies and cryptographic systems we can leverage to design a better system.
    \item Propose and provide a software implementation of a scheme that is minimizes trust requirements while maximizing coverage of the desirable security properties.
    \item Analyse the protocol and the success of the implementation against the criteria used to analyse the existing schemes.
    \item Form a conclusion whether we {\textbf{can}} reach a minimum level of trust while maintaining the desirable security properties.
\end{enumerate}


\section{Structure of the Report and Methodology}
\label{ch:intro:method}

The is structured as shown in the following diagram:

\todo{
    Structure could probably do with a graphic, I like the tree (DAG) diagram
}

In order to compile this report\dots

\todo{ what did I do, literature search, find protocols, security properties and then design a system with the best of each? Oh and build all the stuff in software}



