
\chapter{Distributed Ledger Technology (a.k.a Blockchains)}
\label{ch:blockchain}

\section{Introduction to Blockchains and the Problems they can solve}
\label{ch:blockchain:intro}

Blockchains are fundamentally an immutable ledger: they allow data to be recorded in an append-only manner and such that the order and content of the ledger cannot be disputed in the future. Combined with a common, shared set of rules ---known as \emph{consensus}--- for how to add data to the chain and validate those data, a blockchain allows for mutually untrusting parties to agree on the content and state of a shared immutable ledger.

This forms the basis of cryptocurrencies where the transactions in the currency are entries into this ledger. The transactions themselves use other cryptographic functions to ensure their own rules ---which all parties can agree on--- and the chain is what allows a public system where no node is required to trust any other node, yet all nodes can agree on the content of the ledger.

I believe it was this lack of trust requirements that really allowed \gls{dlt} to become popular. The first well known use of a blockchain was in the cryptocurrency Bitcoin which has gone from interesting experiment to a global buzzword. One of the key factors in its success was the lack of central authority. There was no way to regulate trade in the coins and no restrictions on who could send or receive them. This would not have been possible without the trust-less distributed immutable ledger provided by the blockchain backbone of Bitcoin.

Blockchains can also be used without the public consensus rules in what is known as a Permissioned Blockchain. In this case the data are public, but only authorized entities may write to it. The key feature here is that the public data are tamper-evident. Such systems can be used as public audit logs with a high trust from the public as entries cannot be retroactively altered.

Whilst cryptocurrencies were the first major application of blockchain technology there are other problems \needcite{KM suggests refs} that can benefit from the decentralized nature and immutability. The two properties together imply a form of censorship resistance that centralized systems can never have --- that is, there is no authority that can intercede to perform the censorship, every node in the network has equal rites. Two examples where this is a desirable property are  the attempts to create decentralized alternatives to the most common \gls{dns} used on the Internet \cite{Namecoin,ENSEthereumName,UnstoppableDomainsDocumentation} and the central \gls{pki} root(s) of trust used in many Operating Systems for certificate validation in \gls{tls} \cite{yakubovBlockchainbasedPKIManagement2018,kubilayCertLedgerNewPKI2019,matsumotoIKPTurningPKI2016}.

Blockchains have a major disadvantage to centralized ledgers in that the speed of transactions is often prohibitively slow for many applications. Using the Ethereum blockchain network as an example, at the time of writing the largest transaction volume in a single day was 1,406,016 on Thursday, September 17 2020 \footnote{Ethereum transaction rate data from \surl{https://etherscan.io}} which equates to ~16.2 transactions per second compared to the Visa payment network which claims greater than 56,000 messages per second \footnote{Visa transaction rates as advertised at \surl{https://visanet.co.uk}}.


\section{Structure of a Blockchain}
\label{ch:blockchain:structure}

\begin{figure}[h]
    \centering
    \includesvg[width=\columnwidth]{figures/blockchain-chaining}
    \caption{Basic Blockchain Structure}
    \label{fig:basic-blockchain}
\end{figure}


Blockchains are at their core built from two simple principles. These two principles provide a foundation of the technology and number of useful properties. They are:

\begin{enumerate}
    \item Every block is linked to the previous block in a deterministic manner.
    \item Every honest node in the network must be able to agree on the same chain.
\end{enumerate}

Let us look at the first principle. The important word is \emph{deterministic}. Consider \autoref{fig:basic-blockchain}; each block has its identifier as the output of a cryptographic hash function over data from the payload of the block as well as the hash of the previous block. This way we cannot retrospectively change the contents of a block, as its hash would no longer validate, nor would the hash of any subsequent blocks. To rewrite a block in the chain, we would need to rewrite the entire thing from that block onwards, so that all the new hashes would be correct.

The second principle ensures that while this may not be impossible in principle there must be a method of picking the \emph{true} chain that all honest nodes in the network can agree on --- despite the possible presence of adversarial nodes. This is a property is known as consensus. Firstly, nodes take the view that the longest valid chain is the true chain. The second part varies among chains (discussed later in \autoref{ch:blockchain:structure:consensus}) but the original and most common method is a \gls{pow} scheme. This prevents re-writing the chain by enforcing that to create a valid block, the \emph{miner} must perform some non-negligible amount of computation, requiring time and energy. In order to rewrite a block, the cost in time and energy would need to be paid again for that block and for all subsequent blocks. The consensus algorithm makes it increasingly difficult to change the data in a block the deeper it gets into the chain and at some point the data becomes almost certainly immutable.

\subsection{Blocks and Hashes}
\label{ch:blockchain:structure:basics}

Blockchains use cryptographic hash functions for a number of their properties. Here is a quick overview of the desirable security properties of a cryptographic hash function $H$ \cite{martinEverydayCryptography2017}:

\begin{enumerate}
    \item \textbf{Preimage Resistance} Given a hash output $h$, it should be hard to find a message $m$ such that $H(m) = h$.
    \item \textbf{Second Preimage Resistance} Given a message $m_1$ and the output $H(m_1) = h$ it should be hard to find a message $m_2$ such that $H(m_2) = H(m_1) \equiv h$.
    \item \textbf{Collision Resistance} It should be hard to find two messages $m_1$ and $m_2$ such that $H(m_1) = H(m_2)$.
\end{enumerate}

A fourth useful property which implies and is implied by second preimage resistance is \textbf{Pseudo-Randomness}. This property is satisfied if the output from the function is hard to distinguish from a random oracle. That is, there should be no discernable relation between the input and output and similar inputs should not produce similar outputs. Without this property, second preimage resistance could not be satisfied as we would gain some knowledge of $m_2$ due to its relationship with $m_1$.

Note that second preimage resistance is implied by collision resistance, however it is subtly different as we are not attempting to find \emph{any} collision, but a specific collision for a known message. The recent SHAttered \cite{stevensFirstCollisionFull2017} attack on the SHA1 hash function was used to break collision resistance, not second preimage resistance. Blockchains use cryptographic hashes to link blocks together precisely because of second preimage resistance, which implies that we cannot feasibly forge a block while keeping the hash identical --- which would be the only way to insert a block without having to rewrite the entire following chain.

We see in \autoref{fig:basic-blockchain} the data we want to ensure are immutable ---the block payload, the previous block's hash and metadata--- are included in the block's own hash.  We can validate the block has not been tampered with by recomputing the hash and checking it matches the hash we expect. This also means we cannot change any of the block data without also changing the hash and if we change the hash then the next block will have the incorrect previous hash and the chain is broken.

There is a single exception for the first block in the chain ---known as a \emph{genesis} block--- which by definition cannot have a previous block to reference. Most blockchain implementations prefer to keep the genesis block as similar to any other block as possible and simply use a zero value for the previous block hash.

While this concept is simple, in practice it only defines a valid chain. It is quite possible to create a group of blocks that form a directed acyclic graph rather than a single linear chain. In this situation how do we decide which leaf is the ``true'' chain. This uncertainty leads directly to the requirement for the consensus covered next.

\subsection{Consensus}
\label{ch:blockchain:structure:consensus}

Consensus in a distributed system is the property that the honest nodes in the system will agree about the data in that system, even in the presence of adversarial nodes. This is the reason for the second principle mentioned in \autoref{ch:blockchain:intro} and the reader may have already drawn a parallel to the Byzantine General's Problem \cite{lamportByzantineGeneralsProblem1982} which describes an equivalent problem.

A blockchain peer-to-peer network consists of a group of nodes in a gossip network. Each node searches for other nodes and then \emph{gossips} with them. They swap knowledge of other nodes in the network to expand the number of peers they can attempt to connect to. They also swap knowledge of their view of the blockchain. If one node claims to know a longer chain it will tell its peers, and send the blocks to them. The receiving node must then validate that the blocks are valid. The node will use the consensus algorithm to check this. If the blocks are invalid, then the node will block the sender as a dishonest node, ensuring that eventually dishonest nodes are ejected from the network. If the blocks are valid, then the node can be sure that the creator of the block has indeed put in the cost to create it.

As more than one new block could be sent to a node at the same time, we need a second part to the consensus algorithm to ensure that the chain doesn't split into many chains. This is actually very simple and all blockchains I could find treat the longest valid chain as the \emph{true} chain. If a peer node can produce a longer valid chain, then the network will treat that as the true chain. This happens with a reasonable probability and most blockchains do not consider blocks confirmed unless they have a number (six in the case of Bitcoin) of blocks on ahead of them. After the blocks have been added it is considered infeasible for an adversary to be able to produce a longer valid chain with that block altered.

The consensus method most often used in blockchains is known as a \gls{pow} and involves performing a computational task that takes a non-trivial amount of processing power to perform but a trivial amount to check it was performed correctly. The value invested is the computation work performed.

An early example of a \gls{pow} system is HashCash \cite{backHashcashDenialService2002} initially devised as a spam prevention mechanism. By adding a \gls{pow} requirement to email it would no longer be cheaper for spammers to send mail at volume. The main principle behind HashCash is to give a proof that some work has been done, specifically for a given commitment. The commitment is some data we wish to show we knew before we did the work. The \emph{work} involved is the computation time required to perform the hash function. A single hash requires a very small amount of work, so we want the prover to perform a very large number of hashes, however the verifier should not have to do much work at all. This may seem impossible as the pseudo-randomness property means that to be sure the prover has performed the hashes, the verifier would have to perform the same number of hashes.

The solution to this is to lower our requirement of proof from complete certainty to \emph{within a given probability}. Now we can take advantage of the pseudo-randomness of the hash function. As every input produces a random hash, each bit of the hash will be $0$ or $1$ with probability $1/2$. So producing a hash with the first bit $0$ has probability $1/2$. A hash with the first 2 bits $0$ has probability $1/2^2$. For a hash with the first $n$ bits $0$ the probability is $1/2^n$ and will on average take $2^n$ attempts to find. Now we can create our commitment $c$ with a shared method and append some randomness $r$ and calculate the hash $H(c || r) = h$. If the first $n$ bits of the hash are 0, we consider the work done, if not we create a new random value $r$ and try again. This will take on average $2^n$ attempts, but the verifier given $r$ can create the commitment $c$ and confirm that $H(c || r)$ starts with $n$ 0-bits with a single hash. Pseudo-code for this algorithm is shown in \autoref{fig:hashcash}.

\begin{figure}[h]
    \centering
    \begin{lstlisting}[style=ES6, language=ES6]
// right-shift by this amount will leave only the bits
// we want to check are 0
let shift = numberOfBitsInHashOutput - nBitsShouldBeZero

function hashcash_prove (commitment) {
    do {
        // pick a random value for r
        let r = random()
        // until it verifies
    } while (!hashcash_verify(commitment, r))

    return r
}

function hashcash_verify (commitment, r) {
    // concatenate and apply hash function
    let h = hash(commitment || r)
    // after shift the result should be zero
    return (h >> shift == 0)
}
        \end{lstlisting}
    \caption{Pseudo-code for the HashCash algorithm}
    \label{fig:hashcash}
\end{figure}

For a blockchain we can use a similar function using all the data that makes the block identifier hash as the commitment and trying values for the proof until a satisfactory value is found. What makes the value satisfactory will vary, and is determined by the shared rules of the system. In many implementations the \emph{difficulty} is variable to allow the chain to react to keep the rate of addition of blocks roughly constant.

As the blockchain needs new blocks added, performing the work required for \gls{pow} usually results in a reward for the miner, incentivizing miners to continue to add blocks. This rewards good behaviour in the system, but bad actors are not punished.

The second method of consensus is known as \gls{pos} and instead of performing work to ``prove'' the block has been mined correctly, instead the mining is delegated to a committee of stakeholders. This is simplest to explain in the context of a cryptocurrency where the coin itself is assumed to have value. The idea is that holders of the currency ---that is, those with a \emph{stake} in it and motivation to act in it's best interest--- are chosen to mine blocks. The method of choosing varies among algorithms, but it is usually a group of coin holders where the coin is oldest, ties broken by ``lowest hash'' over the transactions which can be effectively considered a random, but deterministic, choice. Those stakeholders must then digitally sign the block data to \emph{vouch} for it. Only the correct stakeholder can sign the requisite data and the algorithm means all can determine who should sign the data, all parties can confirm the block's legitimacy.

Algorithms based on \gls{pos} have the advantage that they can punish bad actors just as they can reward good behaviour. They also do not excessively waste energy performing the \gls{pow} proofs. However, there has been criticism of proposals for \gls{pos} algorithms \needcite{critics of pos} and although some blockchains \needcite{pos hybrid chains} have adopted a hybrid approach requiring a \gls{pow} block (known as a \emph{checkpoint}) at certain points in the chain.

Although consensus is mostly used to cover the core algorithm for ascertaining that a block has been mined \emph{correctly}, I would argue that the rules inherent in the chain for exactly how the data are laid out and what constitutes valid block payloads should be considered part of the consensus, as blocks will be rejected if they don't match these core rules, just as if the main \gls{pow} or \gls{pos} fails to verify. This concept can be extended to cover a third consensus type known as \gls{poa} which creates a ``permissioned'' blockchain. In this consensus mode the proof that an entity may add a block is governed by ownership of a key that allows the block to be added. For example, a digital signature may be created over the block header which only the owner of a secret key could produce. Provided the swarm agrees on the public key to verify this signature, the chain can grow but only a single entity can add blocks. The real difficulty in such a system is to ensure that all nodes agree on the validity of the key(s) to add new blocks. An adversary could claim their key is the correct one and swiftly rewrite the chain, producing a longer chain that the ``real'' one. One of the benefits of a blockchain is its inherent immutability, which could be broken in this situation. As such, a \gls{poa} system more secure when paired with another method such a \gls{pow} to ensure both the immutability of the chain and the eligibility of a specific entity to write to it.

\section{Transactions, Smart Contracts and other chains}
\label{ch:blockchain:types}

Blockchains can have arbitrary data in the block payloads, each chain defines the rules for what is allowed and the significance of the data. Cryptocurrencies based on blockchains typically allow for transactions moving currency around the system. These transactions may be directly from one user to another, or they may have some more complex logic embedded. This logic is known as a \emph{smart contract} and the functionality available in such contracts is defined by the blockchain itself. Some chains have a Turing-complete instruction set for the smart contracts (e.g. Ethereum) and others a more restricted instruction set (e.g. Bitcoin).

Unless the blockchain has only a single transaction model, there will be some form of smart contract involved with every transaction. Transactions can be split into inputs and outputs, where the funds are sourced from the inputs and moved to the outputs. These transactions also form a chain where outputs of one transaction must be used as the input for another transaction. Some cryptocurrencies have rules for the creation of new currency without a previous transaction as a method of increasing the supply of currency available. The Bitcoin network rewards block miners with an amount of bitcoin in an input-less transaction known as a \emph{coinbase} transaction.

In order to spend funds from a previous transaction output the smart contract comes into play. The logic in the contract will determine what needs to be done to use the funds in a new transaction. A simple contract type would be simply that the spender must prove access to the private key matching a given public key in the transaction. A more complex contract might stipulate that $M$ of $N$ keys must be proved in order to release the funds. Both these types of contract can be created on the Bitcoin network. Contracts of theoretically unlimited complexity may be created on the Ethereum network, however there are practicality issues and costs involved with high complexity contracts.

In the Namecoin (\surl{https://namecoin.org}) blockchain while fully functioning as a cryptocurrency ---it started as a fork of the original Bitcoin code--- also has a second type of output for transactions which allows \emph{purchasing} a key-value pair publicized on the chain which then lasts a fixed period of time before it must be renewed. The main idea behind the chain was to provide a censor-resistant platform for registering \gls{dns} records without a central authority. The cost in the currency would deter squatting or mass registering domains and the immutability and decentralized nature would make external censorship or seizure impossible.

These transactions give a blockchain value, either by simply allowing it to function as a cryptocurrency or as a platform for trust-less computing by the execution of arbitrary smart contracts. There are some blockchains that do not aim for this public utility and instead aim for the immutability for audit purposes. By making it infeasible to alter historic records, the blockchain allows entities to publicize the chain for external audit, even if the rules do not allow external entities to write to the chain. Known as a \emph{permissioned} blockchain where participants have different permission levels, the blockchain can still provide value with no cryptocurrency involved.

\section{Trust in DLT and their use in e-Voting}
\label{ch:blockchain:trust}

Trust is important in decentralized systems. With no authority to control access to the system, it is important to understand that nodes must be able to operate \emph{without} trusting each other. All nodes in the system are considered untrusted, but the consensus algorithm will ensure that the rules are followed. This does mean that all honest nodes need to be adhering to the same rules, but the dishonest parties cannot break them or the honest nodes will notice and ignore them.

This does not mean that a blockchain network is implicitly secure in the presence of adversarial nodes. There are known attacks on blockchain networks and their consensus mechanisms \cite{wenAttacksCountermeasuresBlockchains2021a,eyalMajorityNotEnough2014,sapirshteinOptimalSelfishMining2017,karameDoublespendingFastPayments2012,apostolakiHijackingBitcoinRouting2017}. For example any peer to peer network is susceptible to an eclipse attack \cite{heilmanEclipseAttacksBitcoin2015} in which the adversary gains control over which peers a node is able to connect to. As the targetted node can only connect to adversarial nodes, who can all lie about the state of the network, they can keep the target behind the main chain or trick it into accepting a shorter chain of their own devising. Similarly, a Sybil attack \cite{douceurSybilAttack2002} is when the adversary floods the network with adversarial nodes to the point that they wield influence over the network.

Blockchains that use Proof of Work as a consensus mechanism are susceptible to a 51\% attack although \cite{yangEffectiveScheme512019} proposes a possible solution. This attack is realized when an attacker gains more than 50\% of the mining capacity in the blockchain network. This gives the attacker a level of control over the network as they have the highest chance of being able to mine new blocks. In a cryptocurrency ,the control extends to being able to block transactions at will or ensure specific transactions get mined. This centralization of power would likely cause the value of the coin to drop and would therefore possibly be self-defeating, but could be an effective method of destroying trust in a cryptocurrency.

The Eclipse and Sybil attacks are generic to any peer-to-peer network. In terms of trust, we as participants in a blockchain network must trust that the network is sufficiently robust against these attacks and that there exists a sufficient number of honest nodes. In a cryptocurrency, the block miners are financially incentivized to behave correctly which amplifies the network effect. The more popular blockchain networks see the most miners and users and therefore become the most trustworthy.

How might we encourage good behaviour in a voting system without a reward mechanism system? The simple answer is that we do not need to as our requirements from the chain are quite different from that of a cryptocurrency. The requirements we have are that of a public audit log, very similar to a concept of a permissioned blockchain, where only certain entities can write to the chain.

Blockchains have been used in voting systems as we have seen in \autoref{ch:ev:existing}. Open Vote Network \cite{mccorrySmartContractBoardroom2017} is a system running on a smart contract within the Ethereum blockchain. The protocol there is enforced by the smart contract, and indeed all computation is performed there also. This gave the protocol scalability issues, limiting it to just 40 voters. Subsequently, Scalable Open Vote Network \cite{seifelnasrScalableOpenVoteNetwork2020} addresses the scalability issue by perform most computation off-chain and publishing evidence to the smart contract, which will then allow voters to contest the computation as they see fit. These two systems highlight the high cost of computation on a blockchain in a smart contract.

The Bitcoin Voting \cite{zhaoHowVotePrivately2016} system works with clever use of the Bitcoin transaction system and its smart contract system. While the smart contracts available in Bitcoin are not Turing-complete and in fact are extremely limited, they are still flexible enough to cover the needs of this voting system. However, in this system again, the cost of voting is non-trivial.

Both these previous paragraphs show systems that use blockchain-backed cryptocurrencies as a basis for their protocol. In my opinion the real financial cost of using such a system is prohibitive of it being viable as a national-level voting scheme. However, cryptocurrencies and smart contracts are not the only way to incorporate \gls{dlt} into a voting system. The system proposed in \cite{yangBlockchainVotingPublicly2020} uses the blockchain purely a public, immutable ledger acting as the Bulletin Board aspect of a general voting system. The cryptographic techniques used in the data stored in the blocks ensures the privacy and correctness properties required by the system while keeping all data in the open.
