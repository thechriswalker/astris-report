
\chapter{Blockchain Technology}
\label{ch:blockchain}

\section{Introduction to Blockchains and the Problems they can solve}
\label{ch:blockchain:intro}

\todo{
  Why have blockchains become a thing? What makes them special? \\
  What is the key advancement that has shot them to notoriety? \\
  That is the distributed consensus and immutability. \\
}

\section{Structure of a Blockchain}
\label{ch:blockchain:structure}

Blockchains are at their core built from two simple principles. These two principles provide a foundation of the technology and number of useful properties. They are:

\begin{enumerate}
  \item Every block is linked unforgeably to the previous block.
  \item Creating a block requires some "work".
\end{enumerate}

Let us look at the first principle. The important word is \emph{unforgeably}. Each block has it's contents assured by a cryptographic hash function. This hash includes data from the hash of the previous block. This way we cannot retrospectively change the contents of a block, as it's hash would no longer validate, nor would the hash of any subsequent blocks. This means that to rewrite a block in the chain, we would need to rewrite the entire thing from that block onwards.

\todo {diagram?}

This would not be difficult to do if it weren't for the second principle which says that in order to create (\emph{mine}) a block, we must perform a non-trivial amount of work. This means that in order to rewrite a block, you would need to be able to perform this work again for all the block and for all subsequent blocks.

This immutability of data is the main property of a blockchain. On top of this, a distributed system using a blockchain can agree to a consensus algorithm to allow untrusted systems to agree on the data. We will discuss this in the Consensus section.

\subsection{Blocks, Hashes and Work functions}
\label{ch:blockchain:structure:basics}

Here is a representation of data in a simple block chain. Note that the structure of the most famous blockchain (as used in BitCoin \needcite{bitcoin})

\todo{
  SVG of a few blocks each with a payload, previous hash, proof of work, payload \\
  Show how the block's hash is H(prev, H(payload)) \\
  Do not elaborate on the proof of work here.
}

Each block include a reference to the hash of the previous block. All immutable data is included in the block's own hash, the


\todo{
  How a block is structured, how the hashes are "chained" to the contents and how the blocks can be "validated" back to the "genesis" block. \\
  Maybe Work functions should get their own section as there are options: Proof of Work or Proof of Stake or someother proof. Each option has variants and will be implemented differently. \\
  Look at proof of work and the "hashcash" algorithm, the almost "hashcash" in bitcoin and the PoW in Monero (much more variable). \\
}

\subsection{Consensus}
\label{ch:blockchain:structure:consensus}

\todo{What does consensus mean here, why is it so important? Do different chains use different algorithms then "longest" wins?}

\section{Types of Chains, Transactions and Smart Contracts}
\label{ch:blockchain:types}

\todo{
  Public chains like cryptocurrencies, private chains like internal audit logs. Hybrid distributed ledgers for open viewing, but private updates. \\
  - Bitcoin and transactions \\
  - Ethereum and smart contracts
}


\section{Trust in Blockchains}
\label{ch:blockchain:trust}

\todo{
  How do blockchains reduce trust requirements? What to we need to trust? How is that different/better/worse than centralised systems?
}
