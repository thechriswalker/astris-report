
\chapter{Blockchain Technology}
\label{ch:blockchain}

\section{Introduction to Blockchains and the Problems they can solve}
\label{ch:blockchain:intro}

\todo{
    Why have blockchains become a thing? What makes them special? \\
    What is the key advancement that has shot them to notoriety? \\
    That is the distributed consensus and immutability. \\
}

\section{Structure of a Blockchain}
\label{ch:blockchain:structure}

\todo{Simple tech write up of the basic idea of a blockchain}

\subsection{Blocks, Hashes and Work functions}
\label{ch:blockchain:structure:basics}

\todo{
    How a block is structured, how the hashes are "chained" to the contents and how the blocks can be "validated" back to the "genesis" block. \\
    Maybe Work functions should get their own section as there are options: Proof of Work or Proof of Stake or someother proof. Each option has variants and will be implemented differently. \\
    Look at proof of work and the "hashcash" algorithm, the almost "hashcash" in bitcoin and the PoW in Monero (much more variable). \\
}

\subsection{Consensus}
\label{ch:blockchain:structure:consensus}

\todo{What does consensus mean here, why is it so important? Do different chains use different algorithms then "longest" wins?}

\section{Types of Chains, Transactions and Smart Contracts}
\label{ch:blockchain:types}

\todo{
    Public chains like cryptocurrencies, private chains like internal audit logs. Hybrid distributed ledgers for open viewing, but private updates. \\
    - Bitcoin and transactions \\
    - Ethereum and smart contracts
}


\section{Trust in Blockchains}
\label{ch:blockchain:trust}

\todo{
    How do blockchains reduce trust requirements? What to we need to trust? How is that different/better/worse than centralised systems?
}
