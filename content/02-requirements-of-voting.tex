
\chapter{Requirements of Voting Systems}
\label{ch:req}

\section{Security Requirements}
\label{ch:req:sec}

There are a number of desirable properties for a voting system to have, some of which could be considered required for a voting system given the assumption that the system aspires to be fair and honest. There have been many papers outlining the security requirements specific to voting \cite{epsteinElectronicVoting2007,delauneFormalisingSecurityProperties2010,liTaxonomyComparisonRemote2014,hastingsSecurityConsiderationsRemote2011}, and they tend to agree on the majority. First we make some definitions of the primary entities involved:

\begin{table}[h!]
    \centering
    \begin{tabular}{|Sr|S{p{0.666\textwidth}}|}
        \hline
        \textbf{Entity}         & \textbf{Definition}                                                                                                                                                                                                                                                                                                                                                                                                                         \\
        \hline\hline
        \textbf{Candidate}      & An entity that is entering into the election as a potential choice for the voters. All possible candidates will be listed for the voter to choose from when voting.                                                                                                                                                                                                                                                                         \\
        \hline
        \textbf{Voter}          & An entity that is eligible to vote in the election. Proof of eligibility is not defined here but must be possible within the framework of an election.                                                                                                                                                                                                                                                                                      \\
        \hline
        \textbf{Authority}      & The entity holding the election. The authority will define the parameters of the election, for example which candidates there will be, the timeframe voting will take place in and the voting scheme that will be used.                                                                                                                                                                                                                     \\
        \hline
        \textbf{Registrar}      & An entity that will be held responsible for authenticating the voters, and hence determining eligibility. This entity may be the same as the authority, or it may be separate.                                                                                                                                                                                                                                                              \\
        \hline
        \textbf{Bulletin Board} & Refers to a publicly available record of the information regarding the election. Depending on the voting scheme, this is not always made available and may have differing amounts of information \needcite{no bulletin board}. If it is made available, it will almost certainly be read-only for the public and maintained by the authority. This need not be electronic, or can be even if the voting scheme is otherwise non-electronic. \\
        \hline
        \textbf{Auditor}        & Any entity that validates the election data is correct by inspecting the bulletin board, votes or tallies. In some schemes this is a separate entity \needcite{no public audit} and in others \needcite{public-audit} it can be performed by anyone.                                                                                                                                                                                        \\
        \hline
        \textbf{Adversary}      & Any entity that is attempting to subvert the election in any way. This could be to attempt to have their preferred candidate elected or to stop the election from taking place at all.                                                                                                                                                                                                                                                      \\
        \hline
    \end{tabular}
    \caption{Table of Definitions of Entities in Voting Systems}
    \label{table:voting-entities}
\end{table}

The following properties are basic requirements for any voting system to have practical use:

\begin{description}
    \item[Correctness] The outcome of the election should be a reflection of the votes cast. That is, if the protocol of the voting system is followed and the entities involved are honest to the degree required by the system then the outcome will be a true reflection of the votes cast.
    \item[Secrecy] After casting a vote, no one can learn the choice of the voter.
    \item[Eligibility] Only eligible voters should be able to cast a vote. The conditions for eligibility can vary, but these conditions must be met by all voters and the system must enforce this.
    \item[Fairness] No information about the results --- full or partial tallies --- can be made available before the close of the election. That is, no party can gain information about the tally before all votes have been cast.
\end{description}

The next set of properties are desirable, and indeed should be present in any secure voting system.

\begin{description}
    \item[Robustness] The scheme should be resilient against an adversary attempting to influence the outcome. The influence could be arbitrary, simply attempting to create errors in the process or to create a denial of service stopping the election from happening at all. The influence could be specific, attempting to sway the results in a pre-determined way. The scheme must also be resilient to errors occurring for any other reason; malignant or benign.

    \item[Receipt-Freeness] After casting a vote, the voter should not be able to prove their choices to any entity. This prevents the voter from being able to prove their vote to a 3rd party in a vote buying or selling scheme, or in the case of coercion. This property is similar to \emph{secrecy} but stronger, as not even the voter can prove how they voted.
    \item[Coercion-Resistance] The scheme should prevent an adversary from being able to manipulate a voter into either abstention, or to coerce the voter into voting a certain way. The adversary should not be able to impersonate the user and place a vote on the voter's behalf.
    \item[Individual Verifiability] Any vote may verify that their vote has be counted in the tally.
    \item[Universal Verifiability] Anyone may verify that the final outcome is calculated correctly. The scheme provides enough public evidence to prove the truth of the outcome.
    \item[Auditability] The scheme provides evidence of its behaviour before, during and after an election.
\end{description}

\section{Trust Requirements}
\label{ch:req:trust}

The Oxford English Dictionary defines trust as follows:

\vspace{1em}
\noindent \begin{tabular}{|p{0.9\textwidth}}
    \noindent \textbf{noun}: Firm belief in the reliability, truth, or ability of someone or something.
\end{tabular}
\vspace{1em}

This is a difficult thing to quantify. Romano \cite{romanoNatureTrustConceptual2003} defines a scale for measuring trust, but it is very subjective. We want to remove as much subjectivity from the question of trust and move it into something definite that we can answer --- is it \emph{easier} to trust $X$ or $Y$?

\begin{description}
    \item[Trust can be considered on a binary scale] either you trust something or you don't.
    \item[Trust can be considered on a continuous scale] you can trust something a little or a lot.
\end{description}

I believe the two concepts are related, in that the amount of continuous trust you have in something must exceed the trust requirements it imposes on you for you to consider that you have the binary trust in it.

Often the thing you have continuous trust in --- to whatever extent --- will only be converted into the binary trust at the point of deciding to rely on that thing. At this point the value of the thing and the impact of the trust being subverted come into play to make the final decision.

With respect to voting systems trust is the belief that the system will work fairly and correctly and that it will be resistant to attempts to subvert it. Whether we trust this is the case or not for any given election is subjective and personal, however knowledge that the system is designed in a way to resist subversion attempts will greatly reduce the amount of trust we must extend to the system in order to maintain that belief.

Let us take a simple example based on a well known trust exercise. You stand on a chair and there are $N$ people behind you. You fall back trusting that they will catch you and let us say that if any single person reaches to catch you then you will be safe. Clearly, if you know the people behind you, you will be better placed to make a judgement on each one's individual likelihood of letting you fall. Assuming you have no prior knowledge of the individuals behind you cannot make such a judgement. We do know $N$ and the greater the value of $N$ --- the number of people behind you --- the more likely it is that one of them will act in good faith. We can make our own judgement on the probability of a single person being a bad actor and call it $P_b$ then the probability of all actors being bad --- our failure condition --- will be $(P_b)^N$. For any value of $P_b < 1$ then $(P_b)^N$ becomes small very quickly as $N$ rises.

As we increase the $N$ value of the system of this exercise, we lower the trust requirements. We could make a reasonable assumption that in most cases the individuals will likely not have motivation to subvert the exercise --- people generally have enough empathy and would not want to see anyone fall --- therefore we would estimate our $P_b$ to be low and so even with only a few people we could easily exceed the trust requirements placed on us and fall back confidently.

It seems likely that this concept can extrapolate in most situations. That is to say, where trust is placed in a single entity, spreading that trust amongst a group and only requiring a subset to be trustworthy is likely to lower the overall trust requirements.

In the context of a voting scheme there will always be some requirements for trust, the biggest being that the Registrar confirms eligibility and therefore is able to deny eligibility as well.

\todo{ Now look at the requirements/properties previously stated and discuss trust }