
\chapter{Requirements of Voting Systems}
\label{ch:req}

\section{Security Requirements}
\label{ch:req:sec}

There are a number of desirable properties for a voting system to have, some of which could be considered required for a voting system given the assumption that the system aspires to be fair and honest. There have been many papers outlining the security requirements specific to voting and e-voting \cite{epsteinElectronicVoting2007,delauneFormalisingSecurityProperties2010,liTaxonomyComparisonRemote2014,hastingsSecurityConsiderationsRemote2011}, and they tend to agree on the majority. First we make some definitions of the primary entities involved:

\begin{table}[H]
    \centering
    \begin{tabular}{|Sr S{p{0.666\textwidth}}|}
        \hline
        \textbf{Entity}         & \textbf{Definition}                                                                                                                                                                                                                                                                             \\
        \hline\hline
        \textbf{Candidate}      & An entity that is entering into the election as a potential choice for the voters. All possible candidates will be listed for the voter to choose from when voting.                                                                                                                             \\
        \hline
        \textbf{Voter}          & An entity that is eligible to vote in the election. Proof of eligibility is not defined here but must be possible within the framework of an election.                                                                                                                                          \\
        \hline
        \textbf{Authority}      & The entity holding the election. The authority will define the parameters of the election, for example which candidates there will be, the timeframe voting will take place in and the voting scheme that will be used.                                                                         \\
        \hline
        \textbf{Registrar}      & An entity that will be held responsible for authenticating the voters, and hence determining eligibility. This entity may be the same as the authority, or it may be separate.                                                                                                                  \\
        \hline
        \textbf{Trustee}        & An entity or one of a group of entities responsible for counting the votes and computing the final tally. This may be the original Authority or an external entity.                                                                                                                             \\

        \hline
        \textbf{Bulletin Board} & Refers to a publicly available record of the information regarding the election. Depending on the voting scheme, this is not always made available and may have differing amounts of information. This need not be electronic, or can be even if the voting scheme is otherwise non-electronic. \\

        \hline
        \textbf{Auditor}        & Any entity that validates the election data is correct by inspecting the bulletin board, votes or tallies. In some schemes this is a separate, designated entity or group of entities and in others it can be performed by anyone.                                                              \\

        \hline
        \textbf{Adversary}      & Any entity that is attempting to subvert the election in any way. This could be to attempt to have their preferred candidate elected or to stop the election from taking place at all.                                                                                                          \\
        \hline
    \end{tabular}
    \caption{Table of Definitions of Entities in Voting Systems}
    \label{table:voting-entities}
\end{table}

The following properties are basic requirements for any voting system to have practical use:

\begin{description}
    \item[Correctness] The outcome of the election should be a reflection of the votes cast. That is, if the protocol of the voting system is followed and the entities involved are honest to the degree required by the system then the outcome will be a true reflection of the votes cast.
    \item[Secrecy] After casting a vote, no one can learn the choice of the voter.
    \item[Eligibility] Only eligible voters should be able to cast a vote. The conditions for eligibility can vary, but these conditions must be met by all voters and the system must enforce this.
    \item[Fairness] No information about the results ---full or partial tallies--- can be made available before the close of the election. That is, no party can gain information about the tally before all votes have been cast.
\end{description}

Of these properties, correctness and eligibility appear to me as self-evidently required for a voting system to work at all. If the result is not correct, then the system has failed. If ineligible entities are allowed to cast votes alongside the eligible voters then the result will be skewed.

The other two might not have such a fundamental importance, but I would still argue that they are required for a voting system to be practical. If information about the tally is revealed during the voting process it may give more power to the later voters, who would then be able to vote \emph{tactically} to sway a vote. As an example we consider an election of three candidates: Alice, Bob and Charlie. Say that after 90\% of votes have been cast, we know that Alice and Bob both have 40\% of the vote and Charlie the remaining 20\%. In this situation supporters of Charlie may decide that there is no way to win, and instead of voting for their preferred candidate, they instead vote for the \emph{lesser of two evils}: attempting to prevent their \emph{least} preferred candidate from winning. Therefore, the election would be skewed compared to one where all voters have \emph{the same information} at the point of casting a vote.

Secrecy is an important requirement as it allows voters to cast votes as they wish without concern for the allegiance implied by the vote cast. Voters would be able to vote a certain way to let others know, for prestige, rather than because they think the candidate is the correct choice. Conversely, voters would be able to vote a certain way for fear of being persecuted for their real opinion. That is, secrecy is not such much a requirement for the election process as it is a requirement for voters to feel comfortable enough to use the system as intended --- this ties in with the correctness property, that the vote reflects the voters' will. Without voters willing to use a system, or that system unable to represent their true will, it is of no practical value.

The next set of properties are desirable, and indeed should be present in any secure voting system.

\begin{description}
    \item[Robustness] The scheme should be resilient against an adversary attempting to influence the outcome. The influence could be arbitrary, simply attempting to create errors in the process or to create a denial of service stopping the election from happening at all. The influence could be specific, attempting to sway the results in a pre-determined way. The scheme must also be resilient to errors occurring for any other reason; malignant or benign.
    \item[Receipt-Freeness] After casting a vote, the voter should not be able to prove their choices to any entity. This prevents the voter from being able to prove their vote to a 3rd party in a vote buying or selling scheme, or in the case of coercion. This property is similar to \emph{secrecy} but stronger, as not even the voter can prove how they voted.
    \item[Coercion-Resistance] The scheme should prevent an adversary from being able to manipulate a voter into either abstention, or to coerce the voter into voting a certain way. The adversary should not be able to impersonate the user and place a vote on the voter's behalf.
    \item[Individual Verifiability] Any voter may verify that their vote has be counted in the tally.
    \item[Universal Verifiability] Anyone may verify that the final outcome is calculated correctly from the votes. The scheme provides enough public evidence to prove the truth of the outcome.
    \item[Auditability] The scheme provides evidence of its behaviour during and after an election. That is the scheme provides enough public evidence throughout the course of the election that an auditor may discern any dishonest actions.
\end{description}

Robustness in a voting protocol can prevent continual postponement via general \gls{dos} or even election bias in the case of targetted \gls{dos}. An adversary attempting to prevent the election taking place could, if the process was not robust, could sabotage the process to the extent that the result was invalidated. This could serve the candidate (or supporter of the candidate) that would take the effective outcome of the election should the process fail --- e.g. the current elected candidate may remain in power should the election process fail. Similarly, if the will of a group of voters is supposed to be statistically biased towards a specific candidate (perhaps as a result of opinion polls) then an adversary might perform a targetted \gls{dos} attack on those voters attempting to prevent their votes from being cast. The outcome if the attack was successful would be that the group's preferred candidate would be affected most by the lost votes.

Receipt-Freeness has crossover with the secrecy property, except in this case we want the voter to be \emph{unable} to prove who they voted for, rather than for an outside observer to be able to learn who they voted for. This means that the temptation to vote a certain way to be able to show people you voted that way, in an attempt to curry favour or the like, is removed as there is no way to show how you voted. It also helps towards coercion-resistance as the voter is unable to prove to a third party the way they voted. Similarly, vote buying schemes become more difficult as it becomes impossible to prove to the buyer the way in which you voted. Coercion-resistance takes this a step further to mandate that a voter is not able to be coerced to vote in a specific way. That implies receipt-freeness but the converse is not true as we will see when we get to e-voting specifically (in \autoref{ch:ev:specific}). The reason that coercion-resistance is desirable is that without it, the correctness property, if taken literally as the result representing the voters' will, is at risk.

Verifiability is desirable at it will increase voter confidence in the system. Individual verifiability allows voters to be confident that their vote has been cast as they intended and that their vote has been included in the final tally. Without this assurance, the election and its result is subject to the electorate trusting all the entities involved in the process. Once their vote is cast, they would have no way to check anything and must simply hope that their vote has been cast and counted correctly. Universal verifiability is a stronger property that ensures there is enough public evidence for anyone (including ineligible entities) to check all the votes and that they were counted correctly. This level of transparency again aids in increase confidence in the correct outcome of the election. Auditability is the final step along this path, ensuring that the level of transparency before, during and after the election is such that the outcome cannot be contested and the honesty of the participant entities confirmed.

Auditability is a stronger property than universal verifiability, and requires that the system can not only prove that is has behaved correctly, but also that is has not behaved dis-honestly. For example universal verifiability only requires that all the published votes are tallied correctly but does not enforce that we can detect if votes are missing or have been added post facto, such as in the case of ballot-stuffing. Auditability requires we can tell the difference between a vote cast correctly and one added to fill a gap from an absentee voter. It also requires that we can tell if any tampering occurs such as removing cast votes from the published results.

\section{Trust Requirements}
\label{ch:req:trust}

The Oxford English Dictionary defines trust as follows:

\vspace{1em}
\noindent \begin{tabular}{|p{0.9\textwidth}}
    \noindent \textbf{noun}: Firm belief in the reliability, truth, or ability of someone or something.
\end{tabular}
\vspace{1em}

This is a difficult thing to quantify. Romano \cite{romanoNatureTrustConceptual2003} defines a scale for measuring trust, but it is very subjective. We want to remove as much subjectivity from the question of trust and move it into something definite that we can answer --- is it \emph{easier} to trust $X$ or $Y$?

\begin{description}
    \item[Trust can be considered on a binary scale] either you trust something or you don't.
    \item[Trust can be considered on a continuous scale] you can trust something a little or a lot.
\end{description}

I believe the two concepts are related, in that the amount of continuous trust you have in something must exceed the trust requirements it imposes on you for you to consider that you have the binary trust in it.

Often the thing you have continuous trust in ---to whatever extent--- will only be converted into the binary trust at the point of deciding to rely on that thing. At this point the value of the thing and the impact of the trust being subverted come into play to make the final decision.

With respect to voting systems trust is the belief that the system will work fairly and correctly and that it will be resistant to attempts to subvert it. Whether we trust this is the case or not for any given election is subjective and personal, however knowledge that the system is designed in a way to resist subversion attempts will greatly reduce the amount of trust we must extend to the system in order to maintain that belief.

Let us take a simple example based on a well known trust exercise. You stand on a chair and there are $N$ people behind you. You fall back trusting that they will catch you and let us say that if any single person reaches to catch you then you will be safe. Clearly, if you know the people behind you, you will be better placed to make a judgement on each one's individual likelihood of letting you fall. Assuming you have no prior knowledge of the individuals behind you cannot make such a judgement. We do know $N$ and the greater the value of $N$ ---the number of people behind you--- the more likely it is that one of them will act in good faith. We can make our own judgement on the probability of a single person being a bad actor and call it $P_b$ then the probability of all actors being bad ---our failure condition--- will be $(P_b)^N$. For any value of $P_b < 1$ then $(P_b)^N$ becomes small very quickly as $N$ rises.

As we increase the $N$ value of the system of this exercise, we lower the trust requirements. We could make a reasonable assumption that in most cases the individuals will likely not have motivation to subvert the exercise ---people generally have enough empathy and would not want to see anyone fall--- therefore we would estimate our $P_b$ to be low and so even with only a few people we could easily exceed the trust requirements placed on us and fall back confidently.

It seems likely that this concept can extrapolate in most situations. That is to say, where trust is placed in a single entity, spreading that trust amongst a group and only requiring a subset to be trustworthy is likely to lower the overall trust requirements.

In the context of a voting scheme there will always be some requirements for trust, the biggest being that the Registrar confirms eligibility and therefore is able to deny eligibility as well. Furthermore, the desirable properties from \autoref{ch:req:sec} add little to the election process except to enhance the public confidence that the election process is correct and honest. This implies an increase in trust that the system will behave itself, therefore lowering trust requirements for participating.