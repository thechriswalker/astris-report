\newpage
\vspace*{\fill}

\section*{Abstract}

Voting is the core of democracy however the electorate must trust the authorities running any election will act with integrity and that the electorate will not be coerced into voting against their own will.
The amount of trust that must be placed in the hands of the authorities and the feasibility of coercion depends on the methods used to run the election including how the voting is performed and how the ballots are tallied.
With the advancement of technology we now have electronic voting schemes which variously prioritise security, verifiability or other properties.

This paper looks at a the level of trust these schemes require of both the electorate and the authorities and identifies areas where the level of trust can be reduced or eliminated by the considered use of technology.
I propose an e-voting scheme which uses the discussed ideas to minimises the amount of trust required from users and provide a software implementation of such a system.

\vspace*{\fill}
